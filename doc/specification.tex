%%%%%%%%%%%%%%%%%%%%%%%%%%%%%%%%%%%%%%%%%%%%%%%%%%%%%%%%%%%%%%%%%%%%%%%%%%%%%%%%%%%%%%%%%%%%%%%%%%%%%%%%%%%%%
% % % % % % % % % % % % % % % % % % % % % % % % % % % % % % % % % % % % % % % % % % % % % % % % % % % % % % % 
% = = = = = = = = = = = = = = = = = = = = = = = = = = = = = = = = = = = = = = = = = = = = = = = = = = = = = =
%
% This is where the packages are declared

\documentclass[]{article}

\usepackage[english]{babel}
\usepackage[utf8]{inputenc}
\usepackage[margin=1.5in,letterpaper]{geometry} % modify margins
\usepackage{amsmath}
\usepackage{graphicx}
\usepackage[colorinlistoftodos]{todonotes}
\usepackage{hyperref}

\begin{document}

%
% = = = = = = = = = = = = = = = = = = = = = = = = = = = = = = = = = = = = = = = = = = = = = = = = = = = = = =
% % % % % % % % % % % % % % % % % % % % % % % % % % % % % % % % % % % % % % % % % % % % % % % % % % % % % % % 
%%%%%%%%%%%%%%%%%%%%%%%%%%%%%%%%%%%%%%%%%%%%%%%%%%%%%%%%%%%%%%%%%%%%%%%%%%%%%%%%%%%%%%%%%%%%%%%%%%%%%%%%%%%%%



%%%%%%%%%%%%%%%%%%%%%%%%%%%%%%%%%%%%%%%%%%%%%%%%%%%%%%%%%%%%%%%%%%%%%%%%%%%%%%%%%%%%%%%%%%%%%%%%%%%%%%%%%%%%%
% % % % % % % % % % % % % % % % % % % % % % % % % % % % % % % % % % % % % % % % % % % % % % % % % % % % % % % 
% = = = = = = = = = = = = = = = = = = = = = = = = = = = = = = = = = = = = = = = = = = = = = = = = = = = = = =

\title{Java Typestate Checking Implementation Notes}
\author{Xueying Qin}
\date{\small\today}

\maketitle

% = = = = = = = = = = = = = = = = = = = = = = = = = = = = = = = = = = = = = = = = = = = = = = = = = = = = = =
% % % % % % % % % % % % % % % % % % % % % % % % % % % % % % % % % % % % % % % % % % % % % % % % % % % % % % % 
%%%%%%%%%%%%%%%%%%%%%%%%%%%%%%%%%%%%%%%%%%%%%%%%%%%%%%%%%%%%%%%%%%%%%%%%%%%%%%%%%%%%%%%%%%%%%%%%%%%%%%%%%%%%%
%------------------------------------------------------------------------------------------------------------
%------------------------------------------------------------------------------------------------------------
\section{Structure of ExtendJ}
%------------------------------------------------------------------------------------------------------------
%------------------------------------------------------------------------------------------------------------
\subsection{General Information}
Information of ExtendJ, which is the extendable Java compiler we used for this project, can be found at \url{https://extendj.org}. \\[0.2cm]
API documentation can be found at \url{https://extendj.org/doc/}.\\[0.2cm]
ExtendJ syntax reference can be found at \url{http://jastadd.org/web/documentation/reference-manual.php#jadd}.

\subsection{ExtendJ Problem-Reporting Features}
ExtendJ uses bottom-up problem reporting design. \\[0.2cm]
When the compiler is executed, each \texttt{ASTNode} in the Java abstact syntax tree (AST) contributes problems (including warnings and errors) to \texttt{CompilationUnit}. Root of Java AST, which is \texttt{Program}, processes each \texttt{CompilationUnit} to check whether there are problems need to be reported or not.

%------------------------------------------------------------------------------------------------------------
%------------------------------------------------------------------------------------------------------------
\section{Previous Version of Mungo}
%------------------------------------------------------------------------------------------------------------
%------------------------------------------------------------------------------------------------------------
\subsection{General Information}
Project website: \url{http://www.dcs.gla.ac.uk/research/mungo/index.html}. \\[0.2cm]
More information can be found at \url{https://bitbucket.org/abcd-glasgow/mungo}.\\[0.2cm]
Source code of previous version Mungo can be found at \url{https://github.com/Dimitris-Kouzapas/javatypestate}.

\subsection{Code Reuse}
In current version, the grammar and parser for typestate protocol files are adapted with modification from previous version. The algorithm of typestate protocol sematic checking is also from previous version but implemented using ExtendJ bottom-up problem reporting design. Moreover, helper methods for typestate \texttt{ASTNode} are adapted from previous version.

%------------------------------------------------------------------------------------------------------------
%------------------------------------------------------------------------------------------------------------
\section{Typestate Annotation Detecting and Typestate Protocol File Parsing}
%------------------------------------------------------------------------------------------------------------
%------------------------------------------------------------------------------------------------------------
\subsection{Typestate Annotation Detecting}
After Java source code parsed into AST, similar to collecting problems, the special typestate annotations are reported to \texttt{Program} by nodes in AST. If there are detected typestate annotations, before processing Java syntax checking, protocol files need to be located, parsed and varified.

\subsection{Typestate Protocol File Parsing}
A typestate protocol file contains only one Java class typestate declaration. It will be parsed into a \texttt{CompilationUnit} and added to \texttt{Program}. Then a semantic checking on the protocol AST will be performed. Semantic errors of protocol files will also be reported. Only if all the protocol files pass semantic checking, Java source files will be checked. \\[0.2cm]
Notice that a protocol file is parsed as a part of the Java AST. If a traversal from \texttt{Program} is performed, all nodes in the subtree parsed from the protocol file will be visited.

%------------------------------------------------------------------------------------------------------------
%------------------------------------------------------------------------------------------------------------
\section{Storing and Updating Typestate Information}
%------------------------------------------------------------------------------------------------------------
%------------------------------------------------------------------------------------------------------------
In current design, typestate information is stored in two formats. In an \texttt{ASTNode} with typestate specifcation, typestate information is stored as a field. In each \texttt{ClassDecl}, there will be a \texttt{SymbolTable} storing global and local variables' typestates.\\[0.2cm]
The original idea was to only store typestate in field, however this design caused difficulties in nested switch statement checking. Using a \texttt{SymbolTable} to store typestate information is a good solution to this issue.\\[0.2cm]
Although using \texttt{SymbolTable} could handle variable typestate checking and updating, storing information in ASTNode is still necessary in current design. Since some \texttt{ASTNode} with typestate cannot be stored in the \texttt{SymbolTable}. For instance, the return typestate of a \texttt{MethodDecl}, the typestate of a \texttt{Dot} and so on. \\[0.2cm]
Update of typestate state is done during typestate checking, i.e., problems reporting. Although it is not a good design to perform typestate checking and state updating at the same time, it is the best way to implement this type checking under ExtendJ framework.


%
% = = = = = = = = = = = = = = = = = = = = = = = = = = = = = = = = = = = = = = = = = = = = = = = = = = = = = =
% % % % % % % % % % % % % % % % % % % % % % % % % % % % % % % % % % % % % % % % % % % % % % % % % % % % % % % 
%%%%%%%%%%%%%%%%%%%%%%%%%%%%%%%%%%%%%%%%%%%%%%%%%%%%%%%%%%%%%%%%%%%%%%%%%%%%%%%%%%%%%%%%%%%%%%%%%%%%%%%%%%%%%


\end{document}